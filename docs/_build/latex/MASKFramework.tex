%% Generated by Sphinx.
\def\sphinxdocclass{report}
\documentclass[letterpaper,10pt,english]{sphinxmanual}
\ifdefined\pdfpxdimen
   \let\sphinxpxdimen\pdfpxdimen\else\newdimen\sphinxpxdimen
\fi \sphinxpxdimen=.75bp\relax

\PassOptionsToPackage{warn}{textcomp}
\usepackage[utf8]{inputenc}
\ifdefined\DeclareUnicodeCharacter
% support both utf8 and utf8x syntaxes
\edef\sphinxdqmaybe{\ifdefined\DeclareUnicodeCharacterAsOptional\string"\fi}
  \DeclareUnicodeCharacter{\sphinxdqmaybe00A0}{\nobreakspace}
  \DeclareUnicodeCharacter{\sphinxdqmaybe2500}{\sphinxunichar{2500}}
  \DeclareUnicodeCharacter{\sphinxdqmaybe2502}{\sphinxunichar{2502}}
  \DeclareUnicodeCharacter{\sphinxdqmaybe2514}{\sphinxunichar{2514}}
  \DeclareUnicodeCharacter{\sphinxdqmaybe251C}{\sphinxunichar{251C}}
  \DeclareUnicodeCharacter{\sphinxdqmaybe2572}{\textbackslash}
\fi
\usepackage{cmap}
\usepackage[T1]{fontenc}
\usepackage{amsmath,amssymb,amstext}
\usepackage{babel}
\usepackage{times}
\usepackage[Bjarne]{fncychap}
\usepackage{sphinx}

\fvset{fontsize=\small}
\usepackage{geometry}

% Include hyperref last.
\usepackage{hyperref}
% Fix anchor placement for figures with captions.
\usepackage{hypcap}% it must be loaded after hyperref.
% Set up styles of URL: it should be placed after hyperref.
\urlstyle{same}
\addto\captionsenglish{\renewcommand{\contentsname}{Contents:}}

\addto\captionsenglish{\renewcommand{\figurename}{Fig.\@ }}
\makeatletter
\def\fnum@figure{\figurename\thefigure{}}
\makeatother
\addto\captionsenglish{\renewcommand{\tablename}{Table }}
\makeatletter
\def\fnum@table{\tablename\thetable{}}
\makeatother
\addto\captionsenglish{\renewcommand{\literalblockname}{Listing}}

\addto\captionsenglish{\renewcommand{\literalblockcontinuedname}{continued from previous page}}
\addto\captionsenglish{\renewcommand{\literalblockcontinuesname}{continues on next page}}
\addto\captionsenglish{\renewcommand{\sphinxnonalphabeticalgroupname}{Non-alphabetical}}
\addto\captionsenglish{\renewcommand{\sphinxsymbolsname}{Symbols}}
\addto\captionsenglish{\renewcommand{\sphinxnumbersname}{Numbers}}

\addto\extrasenglish{\def\pageautorefname{page}}

\setcounter{tocdepth}{4}
\setcounter{secnumdepth}{4}


\title{MASK Framework}
\date{Jun 18, 2019}
\release{}
\author{}
\newcommand{\sphinxlogo}{\vbox{}}
\renewcommand{\releasename}{}
\makeindex
\begin{document}

\pagestyle{empty}
\sphinxmaketitle
\pagestyle{plain}
\sphinxtableofcontents
\pagestyle{normal}
\phantomsection\label{\detokenize{contents::doc}}



\chapter{Intorduction}
\label{\detokenize{contents:intorduction}}
\sphinxstylestrong{MASK Framework is an open-source framework for de-identification of medical free-text data}

In this project, we will develop an open-source framework for automated de-identification of medical textual data. Such data contains information that can be utilized to support clinical research, but its native form contains sensitive personal identifiable information (PII) that should not be accessed by anyone who does not provide direct clinical care.

The project aims to enhance the current processes and build an open-source platform that can be used for flexible masking of personal information, ensuring that de-identified medical text still contains enough information to facilitate research.

In order to facilitate flexibility, the de-identification system has to be configurable by the user in terms of:
\begin{itemize}
\item {} 
Types of PII that have to be identified in free-text data;

\item {} 
Approaches to masking of the identified data (keep, redact, map, etc.);

\item {} 
Disclosure risk analysis that is performed on the data;

\item {} 
The methodology that is applied for each of the steps.

\end{itemize}


\chapter{Architectural considerations}
\label{\detokenize{contents:architectural-considerations}}

\section{Configuration}
\label{\detokenize{contents:configuration}}
The requirements for the configuration file are:
\begin{itemize}
\item {} 
Store the information about algorithms that should be used for NER

\item {} 
This can be done for per entity

\item {} \begin{description}
\item[{Store information about masking}] \leavevmode\begin{itemize}
\item {} 
Which named entities to mask

\item {} 
How these named entities should be masked

\item {} 
There can be a choice: do not mask, map and redact

\end{itemize}

\end{description}

\item {} 
Talk to ICES what should we implement as examples (name, postcode, age intervals)

\item {} 
User can pick algorithm for mapping

\item {} 
Algorithms for mapping can be added as plugins

\item {} 
Mapping algorithms should be defined for each NER

\end{itemize}


\section{Architectural consideration for extendable framework and configuration}
\label{\detokenize{contents:architectural-consideration-for-extendable-framework-and-configuration}}
For named entity recognition algorithms there are following considerations:
\begin{itemize}
\item {} 
All implementations should be implemented in a single file as a class

\item {} 
All implementations should be stored in a single folder

\item {} 
All implementations should inherit same abstract class, implement method initialize (should load the models), perform\_NER (takes string and returns an array of tuples with class, begin span, end span).

\item {} 
They should all return a subset of defined classes (PATIENT\_NAME, DOCTOR\_NAME, PROFESSION, ADDRESS, CITY, COUNTRY, POST\_CODE, PHONE\_NUMBER, EMAIL, WEB\_ADDRESS, PATIENT\_ID, DOCTOR\_ID, ORGANIZATION, DATE)

\item {} 
Defined functions in the config file should correspond to the class and file names in this directory

\end{itemize}

For extensions related to masking functions there are following considerations:
\begin{itemize}
\item {} 
All implementations should be implemented in a single file as a class

\item {} 
All implementations should be all stored in a single folder

\item {} 
All implementations should inherit the same abstract class and implement “mask” method that takes as input string to be masked and return masked string (either mapped or redacted in a particular manner).

\item {} 
Defined functions in the config file should correspond to the class and file names in this directory

\end{itemize}


\section{Configuration file example and explanation}
\label{\detokenize{contents:configuration-file-example-and-explanation}}
Example of configuration file:

\begin{sphinxVerbatim}[commandchars=\\\{\}]
\PYG{o}{\PYGZlt{}}\PYG{n}{project}\PYG{o}{\PYGZgt{}}
    \PYG{o}{\PYGZlt{}}\PYG{n}{project\PYGZus{}name}\PYG{o}{\PYGZgt{}}\PYG{n}{Masking} \PYG{n}{v1}\PYG{o}{\PYGZlt{}}\PYG{o}{/}\PYG{n}{project\PYGZus{}name}\PYG{o}{\PYGZgt{}}
    \PYG{o}{\PYGZlt{}}\PYG{n}{project\PYGZus{}start\PYGZus{}date}\PYG{o}{\PYGZgt{}}\PYG{l+m+mi}{30}\PYG{o}{/}\PYG{l+m+mi}{05}\PYG{o}{/}\PYG{l+m+mi}{2019}\PYG{o}{\PYGZlt{}}\PYG{o}{/}\PYG{n}{project\PYGZus{}start\PYGZus{}date}\PYG{o}{\PYGZgt{}}
    \PYG{o}{\PYGZlt{}}\PYG{n}{project\PYGZus{}owner}\PYG{o}{\PYGZgt{}}\PYG{n}{Nikola} \PYG{n}{Milosevic}\PYG{o}{\PYGZlt{}}\PYG{o}{/}\PYG{n}{project\PYGZus{}owner}\PYG{o}{\PYGZgt{}}
    \PYG{o}{\PYGZlt{}}\PYG{n}{project\PYGZus{}owner\PYGZus{}contact}\PYG{o}{\PYGZgt{}}\PYG{n}{nikola}\PYG{o}{.}\PYG{n}{milosevic}\PYG{n+nd}{@manchester}\PYG{o}{.}\PYG{n}{ac}\PYG{o}{.}\PYG{n}{uk}\PYG{o}{\PYGZlt{}}\PYG{o}{/}\PYG{n}{project\PYGZus{}owner\PYGZus{}contact}\PYG{o}{\PYGZgt{}}
    \PYG{o}{\PYGZlt{}}\PYG{n}{algorithms}\PYG{o}{\PYGZgt{}}
        \PYG{o}{\PYGZlt{}}\PYG{n}{entity}\PYG{o}{\PYGZgt{}}
            \PYG{o}{\PYGZlt{}}\PYG{n}{entity\PYGZus{}name}\PYG{o}{\PYGZgt{}}\PYG{n}{NAME}\PYG{o}{\PYGZlt{}}\PYG{o}{/}\PYG{n}{entity\PYGZus{}name}\PYG{o}{\PYGZgt{}}
            \PYG{o}{\PYGZlt{}}\PYG{n}{original\PYGZus{}name}\PYG{o}{\PYGZgt{}}\PYG{n}{NAME}\PYG{o}{\PYGZlt{}}\PYG{o}{/}\PYG{n}{original\PYGZus{}name}\PYG{o}{\PYGZgt{}}
            \PYG{o}{\PYGZlt{}}\PYG{n}{algorithm}\PYG{o}{\PYGZgt{}}\PYG{n}{NER\PYGZus{}BiLSTM\PYGZus{}Glove\PYGZus{}i2b2}\PYG{o}{\PYGZlt{}}\PYG{o}{/}\PYG{n}{algorithm}\PYG{o}{\PYGZgt{}}
            \PYG{o}{\PYGZlt{}}\PYG{n}{masking\PYGZus{}type}\PYG{o}{\PYGZgt{}}\PYG{n}{Redact}\PYG{o}{\PYGZlt{}}\PYG{o}{/}\PYG{n}{masking\PYGZus{}type}\PYG{o}{\PYGZgt{}}
        \PYG{o}{\PYGZlt{}}\PYG{o}{/}\PYG{n}{entity}\PYG{o}{\PYGZgt{}}
        \PYG{o}{\PYGZlt{}}\PYG{n}{entity}\PYG{o}{\PYGZgt{}}
            \PYG{o}{\PYGZlt{}}\PYG{n}{entity\PYGZus{}name}\PYG{o}{\PYGZgt{}}\PYG{n}{DATE}\PYG{o}{\PYGZlt{}}\PYG{o}{/}\PYG{n}{entity\PYGZus{}name}\PYG{o}{\PYGZgt{}}
            \PYG{o}{\PYGZlt{}}\PYG{n}{original\PYGZus{}name}\PYG{o}{\PYGZgt{}}\PYG{n}{DATE}\PYG{o}{\PYGZlt{}}\PYG{o}{/}\PYG{n}{original\PYGZus{}name}\PYG{o}{\PYGZgt{}}
            \PYG{o}{\PYGZlt{}}\PYG{n}{algorithm}\PYG{o}{\PYGZgt{}}\PYG{n}{NER\PYGZus{}CRF}\PYG{o}{\PYGZlt{}}\PYG{o}{/}\PYG{n}{algorithm}\PYG{o}{\PYGZgt{}}
            \PYG{o}{\PYGZlt{}}\PYG{n}{masking\PYGZus{}type}\PYG{o}{\PYGZgt{}}\PYG{n}{Mask}\PYG{o}{\PYGZlt{}}\PYG{o}{/}\PYG{n}{masking\PYGZus{}type}\PYG{o}{\PYGZgt{}}
            \PYG{o}{\PYGZlt{}}\PYG{n}{masking\PYGZus{}class}\PYG{o}{\PYGZgt{}}\PYG{n}{Mask\PYGZus{}date\PYGZus{}simple}\PYG{o}{\PYGZlt{}}\PYG{o}{/}\PYG{n}{masking\PYGZus{}class}\PYG{o}{\PYGZgt{}}
        \PYG{o}{\PYGZlt{}}\PYG{o}{/}\PYG{n}{entity}\PYG{o}{\PYGZgt{}}
    \PYG{o}{\PYGZlt{}}\PYG{o}{/}\PYG{n}{algorithms}\PYG{o}{\PYGZgt{}}
    \PYG{o}{\PYGZlt{}}\PYG{n}{dataset}\PYG{o}{\PYGZgt{}}
        \PYG{o}{\PYGZlt{}}\PYG{n}{dataset\PYGZus{}location}\PYG{o}{\PYGZgt{}}\PYG{n}{dataset}\PYG{o}{/}\PYG{n+nb}{input}\PYG{o}{\PYGZlt{}}\PYG{o}{/}\PYG{n}{dataset\PYGZus{}location}\PYG{o}{\PYGZgt{}}
        \PYG{o}{\PYGZlt{}}\PYG{n}{data\PYGZus{}output}\PYG{o}{\PYGZgt{}}\PYG{n}{dataset}\PYG{o}{/}\PYG{n}{output}\PYG{o}{\PYGZlt{}}\PYG{o}{/}\PYG{n}{data\PYGZus{}output}\PYG{o}{\PYGZgt{}}
    \PYG{o}{\PYGZlt{}}\PYG{o}{/}\PYG{n}{dataset}\PYG{o}{\PYGZgt{}}
\PYG{o}{\PYGZlt{}}\PYG{o}{/}\PYG{n}{project}\PYG{o}{\PYGZgt{}}
\end{sphinxVerbatim}


\section{Explanation}
\label{\detokenize{contents:explanation}}
The whole configuration is wrapped in \textless{}project\textgreater{} tag. The user can name the project (using \textless{}project\_name\textgreater{}), and give some basic information about creator and contact details. For each entity, user would like to mask, he/she needs to create \textless{}entity\textgreater{} tag.

Inside \textless{}entity\textgreater{} tag, user has to define entity name (using entity\_name tag), he can specify original name that his named entity recognizer outputs (using original\_name tag), specify NER algorithm for recognition (using \textless{}algorithm\textgreater{} tag) and define masking. Masking can be defined by specifying masking type (using masking\_type tag). Possible values for masking type are:
\begin{itemize}
\item {} 
Nothing - does nothing, does not redact or mask entity, but leaves it in text.

\item {} 
Mask - masks entity with another string. The way of masking has to be defined with the masking\_class tag.

\item {} 
Redact - redacts the entity (setting either XXX or entity name - to be discussed in the future).

\end{itemize}


\chapter{Dependancies}
\label{\detokenize{contents:dependancies}}
The system is implemented in Python, using Python 3.5.2. Dependances are defined in requirements.txt file and can be installed by running:

\begin{sphinxVerbatim}[commandchars=\\\{\}]
pip3 install \PYGZhy{}r requirements.txt
\end{sphinxVerbatim}

List of all requirements:

\begin{sphinxVerbatim}[commandchars=\\\{\}]
\PYG{n}{sklearn\PYGZus{}crfsuite}\PYG{o}{==}\PYG{l+m+mf}{0.3}\PYG{o}{.}\PYG{l+m+mi}{6}
\PYG{n}{nltk}\PYG{o}{==}\PYG{l+m+mf}{3.2}\PYG{o}{.}\PYG{l+m+mi}{5}
\PYG{n}{sphinx}\PYG{o}{==}\PYG{l+m+mf}{1.8}\PYG{o}{.}\PYG{o}{*}
\PYG{n}{sphinx\PYGZus{}rtd\PYGZus{}theme}\PYG{o}{==}\PYG{l+m+mf}{0.4}\PYG{o}{.}\PYG{o}{*}
\end{sphinxVerbatim}


\chapter{Example}
\label{\detokenize{contents:example}}
Input file (with configuration presented in the example configuration):

\begin{sphinxVerbatim}[commandchars=\\\{\}]
\PYG{n}{Record} \PYG{n}{Date}\PYG{p}{:} \PYG{l+m+mi}{2070}\PYG{o}{\PYGZhy{}}\PYG{l+m+mi}{12}\PYG{o}{\PYGZhy{}}\PYG{l+m+mi}{01}

\PYG{n}{Narrative} \PYG{n}{History}	

 \PYG{n}{Patient}  \PYG{n}{presents} \PYG{k}{for} \PYG{n}{an} \PYG{n}{annual} \PYG{n}{exam}\PYG{o}{.}  



\PYG{n}{Seen} \PYG{n}{few} \PYG{n}{weeks} \PYG{n}{ago} \PYG{k}{for} \PYG{n}{hair} \PYG{n}{breaking}\PYG{o}{.}



\PYG{n}{GYN} \PYG{o}{\PYGZhy{}} \PYG{n}{thinks} \PYG{n}{about} \PYG{l+m+mi}{2} \PYG{n}{years} \PYG{n}{since} \PYG{n}{last} \PYG{n}{period}\PYG{o}{.} \PYG{n}{Having} \PYG{n}{some} \PYG{n}{tolerable} \PYG{n}{hot} \PYG{n}{flashes}\PYG{o}{.}  \PYG{n}{Last} \PYG{n}{saw} \PYG{n}{Dr} \PYG{n}{Foust} \PYG{n}{of} \PYG{n}{gyn} \PYG{o+ow}{in} \PYG{l+m+mi}{4}\PYG{o}{/}\PYG{l+m+mi}{66}\PYG{p}{,} \PYG{n}{Pap} \PYG{n}{smear} \PYG{n}{done} \PYG{n}{then}\PYG{o}{.} \PYG{n}{Diff} \PYG{n}{exam} \PYG{n}{secondary} \PYG{n}{to} \PYG{n}{way} \PYG{n}{uterus} \PYG{n}{tipped}\PYG{o}{.}



\PYG{n}{Exercise} \PYG{o}{\PYGZhy{}} \PYG{n}{Started} \PYG{n}{walking} \PYG{n}{at} \PYG{n}{work} \PYG{n}{again} \PYG{n}{daily} \PYG{l+m+mi}{1} \PYG{n}{mile}\PYG{o}{.} \PYG{n}{also} \PYG{n}{watching} \PYG{n}{diet} \PYG{n}{now}\PYG{o}{.}

\end{sphinxVerbatim}

Example output (dates are substituted with DATE and names with XXX):

\begin{sphinxVerbatim}[commandchars=\\\{\}]
\PYG{n}{Record} \PYG{n}{Date}\PYG{p}{:} \PYG{n}{DATE}

\PYG{n}{Narrative} \PYG{n}{History}	

 \PYG{n}{Patient}  \PYG{n}{presents} \PYG{k}{for} \PYG{n}{an} \PYG{n}{annual} \PYG{n}{exam}\PYG{o}{.}  



\PYG{n}{Seen} \PYG{n}{few} \PYG{n}{weeks} \PYG{n}{ago} \PYG{k}{for} \PYG{n}{hair} \PYG{n}{breaking}\PYG{o}{.}



\PYG{n}{GYN} \PYG{o}{\PYGZhy{}} \PYG{n}{thinks} \PYG{n}{about} \PYG{l+m+mi}{2} \PYG{n}{years} \PYG{n}{since} \PYG{n}{last} \PYG{n}{period}\PYG{o}{.} \PYG{n}{Having} \PYG{n}{some} \PYG{n}{tolerable} \PYG{n}{hot} \PYG{n}{flashes}\PYG{o}{.}  \PYG{n}{Last} \PYG{n}{saw} \PYG{n}{Dr} \PYG{n}{XXX} \PYG{n}{of} \PYG{n}{gyn} \PYG{o+ow}{in} \PYG{n}{DATE}\PYG{p}{,} \PYG{n}{Pap} \PYG{n}{smear} \PYG{n}{done} \PYG{n}{then}\PYG{o}{.} \PYG{n}{Diff} \PYG{n}{exam} \PYG{n}{secondary} \PYG{n}{to} \PYG{n}{way} \PYG{n}{uterus} \PYG{n}{tipped}\PYG{o}{.}



\PYG{n}{Exercise} \PYG{o}{\PYGZhy{}} \PYG{n}{Started} \PYG{n}{walking} \PYG{n}{at} \PYG{n}{work} \PYG{n}{again} \PYG{n}{daily} \PYG{l+m+mi}{1} \PYG{n}{mile}\PYG{o}{.} \PYG{n}{also} \PYG{n}{watching} \PYG{n}{diet} \PYG{n}{now}\PYG{o}{.}

\end{sphinxVerbatim}


\chapter{Classes and functions}
\label{\detokenize{contents:module-mask_framework}}\label{\detokenize{contents:classes-and-functions}}\index{mask\_framework (module)@\spxentry{mask\_framework}\spxextra{module}}
\sphinxstyleemphasis{mask\_framework.py} \textendash{} Main MASK Framework module
\index{Configuration (class in mask\_framework)@\spxentry{Configuration}\spxextra{class in mask\_framework}}

\begin{fulllineitems}
\phantomsection\label{\detokenize{contents:mask_framework.Configuration}}\pysiglinewithargsret{\sphinxbfcode{\sphinxupquote{class }}\sphinxcode{\sphinxupquote{mask\_framework.}}\sphinxbfcode{\sphinxupquote{Configuration}}}{\emph{configuration='configuration.cnf'}}{}
Class for reading configuration file

Init function that can take configuration file, or it uses default location: configuration.cnf file in folder
where mask\_framework is

\end{fulllineitems}

\index{consolidate\_NER\_results() (in module mask\_framework)@\spxentry{consolidate\_NER\_results()}\spxextra{in module mask\_framework}}

\begin{fulllineitems}
\phantomsection\label{\detokenize{contents:mask_framework.consolidate_NER_results}}\pysiglinewithargsret{\sphinxcode{\sphinxupquote{mask\_framework.}}\sphinxbfcode{\sphinxupquote{consolidate\_NER\_results}}}{\emph{final\_sequences}, \emph{text}}{}
Function that from a list of sequences returned from the NER function is updated with spans
:param final\_sequences: Sequences returned from NER function. Sequence is a array of arrays of tokens in format (token,label).
:param text: full text article
:return: a list of tuples that includes spans in the following format: (token,label,span\_begin,span\_end)

\end{fulllineitems}

\index{main() (in module mask\_framework)@\spxentry{main()}\spxextra{in module mask\_framework}}

\begin{fulllineitems}
\phantomsection\label{\detokenize{contents:mask_framework.main}}\pysiglinewithargsret{\sphinxcode{\sphinxupquote{mask\_framework.}}\sphinxbfcode{\sphinxupquote{main}}}{}{}
Main MASK Framework function

\end{fulllineitems}

\index{recalculate\_tokens() (in module mask\_framework)@\spxentry{recalculate\_tokens()}\spxextra{in module mask\_framework}}

\begin{fulllineitems}
\phantomsection\label{\detokenize{contents:mask_framework.recalculate_tokens}}\pysiglinewithargsret{\sphinxcode{\sphinxupquote{mask\_framework.}}\sphinxbfcode{\sphinxupquote{recalculate\_tokens}}}{\emph{token\_array}, \emph{index}, \emph{token\_size}, \emph{replacement\_size}, \emph{new\_text}, \emph{new\_token}}{}
Function that recalculates token spans when the token is replaced
\begin{quote}\begin{description}
\item[{Parameters}] \leavevmode\begin{itemize}
\item {} 
\sphinxstyleliteralstrong{\sphinxupquote{token\_array}} \textendash{} Array of tokens with all information, including label and spans

\item {} 
\sphinxstyleliteralstrong{\sphinxupquote{index}} \textendash{} Index of the token in the array that is being replaced

\item {} 
\sphinxstyleliteralstrong{\sphinxupquote{token\_size}} \textendash{} size of the token that is being replaced

\item {} 
\sphinxstyleliteralstrong{\sphinxupquote{replacement\_size}} \textendash{} size of the new token that is replacing token

\item {} 
\sphinxstyleliteralstrong{\sphinxupquote{new\_text}} \textendash{} whole text (have been used for debugging purposes, not obsolete and can be empty string)

\item {} 
\sphinxstyleliteralstrong{\sphinxupquote{new\_token}} \textendash{} New string that is replacing the token.

\end{itemize}

\item[{Returns}] \leavevmode
new, modified list of tokens with information about labels and spans. Basically list of tuples (token,label,start\_span,end\_span)

\end{description}\end{quote}

\end{fulllineitems}

\phantomsection\label{\detokenize{contents:module-ner_plugins}}\index{ner\_plugins (module)@\spxentry{ner\_plugins}\spxextra{module}}
\sphinxstyleemphasis{ner\_plugins} - a set of modules that can perform named entity recognition. Basically, plugins for different kinds of named entity recognition

\phantomsection\label{\detokenize{contents:module-masking_plugins}}\index{masking\_plugins (module)@\spxentry{masking\_plugins}\spxextra{module}}\index{Configuration (class in mask\_framework)@\spxentry{Configuration}\spxextra{class in mask\_framework}}

\begin{fulllineitems}
\pysiglinewithargsret{\sphinxbfcode{\sphinxupquote{class }}\sphinxcode{\sphinxupquote{mask\_framework.}}\sphinxbfcode{\sphinxupquote{Configuration}}}{\emph{configuration='configuration.cnf'}}{}
Class for reading configuration file

Init function that can take configuration file, or it uses default location: configuration.cnf file in folder
where mask\_framework is

\end{fulllineitems}

\index{NER\_abstract (class in ner\_plugins.NER\_abstract)@\spxentry{NER\_abstract}\spxextra{class in ner\_plugins.NER\_abstract}}

\begin{fulllineitems}
\phantomsection\label{\detokenize{contents:ner_plugins.NER_abstract.NER_abstract}}\pysigline{\sphinxbfcode{\sphinxupquote{class }}\sphinxcode{\sphinxupquote{ner\_plugins.NER\_abstract.}}\sphinxbfcode{\sphinxupquote{NER\_abstract}}}
Abstract class that other NER plugins should implement
\index{perform\_NER() (ner\_plugins.NER\_abstract.NER\_abstract method)@\spxentry{perform\_NER()}\spxextra{ner\_plugins.NER\_abstract.NER\_abstract method}}

\begin{fulllineitems}
\phantomsection\label{\detokenize{contents:ner_plugins.NER_abstract.NER_abstract.perform_NER}}\pysiglinewithargsret{\sphinxbfcode{\sphinxupquote{perform\_NER}}}{\emph{text}}{}
Implementation of the method that should perform named entity recognition

\end{fulllineitems}


\end{fulllineitems}

\index{NER\_CRF (class in ner\_plugins.NER\_CRF)@\spxentry{NER\_CRF}\spxextra{class in ner\_plugins.NER\_CRF}}

\begin{fulllineitems}
\phantomsection\label{\detokenize{contents:ner_plugins.NER_CRF.NER_CRF}}\pysigline{\sphinxbfcode{\sphinxupquote{class }}\sphinxcode{\sphinxupquote{ner\_plugins.NER\_CRF.}}\sphinxbfcode{\sphinxupquote{NER\_CRF}}}
The class for executing CRF labelling based on i2b2 dataset (2014).
\index{doc2features() (ner\_plugins.NER\_CRF.NER\_CRF method)@\spxentry{doc2features()}\spxextra{ner\_plugins.NER\_CRF.NER\_CRF method}}

\begin{fulllineitems}
\phantomsection\label{\detokenize{contents:ner_plugins.NER_CRF.NER_CRF.doc2features}}\pysiglinewithargsret{\sphinxbfcode{\sphinxupquote{doc2features}}}{\emph{sent}}{}
Transforms a sentence to a sequence of features
\begin{quote}\begin{description}
\item[{Parameters}] \leavevmode
\sphinxstyleliteralstrong{\sphinxupquote{sent}} \textendash{} a set of tokens that will be transformed to features

\end{description}\end{quote}

\end{fulllineitems}

\index{perform\_NER() (ner\_plugins.NER\_CRF.NER\_CRF method)@\spxentry{perform\_NER()}\spxextra{ner\_plugins.NER\_CRF.NER\_CRF method}}

\begin{fulllineitems}
\phantomsection\label{\detokenize{contents:ner_plugins.NER_CRF.NER_CRF.perform_NER}}\pysiglinewithargsret{\sphinxbfcode{\sphinxupquote{perform\_NER}}}{\emph{text}}{}
Implemented function that performs named entity recognition using CRF. Returns a sequence of tuples (token,label).
\begin{quote}\begin{description}
\item[{Parameters}] \leavevmode
\sphinxstyleliteralstrong{\sphinxupquote{text}} \textendash{} text over which should be performed named entity recognition

\end{description}\end{quote}

\end{fulllineitems}

\index{word2features() (ner\_plugins.NER\_CRF.NER\_CRF method)@\spxentry{word2features()}\spxextra{ner\_plugins.NER\_CRF.NER\_CRF method}}

\begin{fulllineitems}
\phantomsection\label{\detokenize{contents:ner_plugins.NER_CRF.NER_CRF.word2features}}\pysiglinewithargsret{\sphinxbfcode{\sphinxupquote{word2features}}}{\emph{sent}, \emph{i}}{}
Transforms words into features that are fed into CRF model
\begin{quote}\begin{description}
\item[{Parameters}] \leavevmode\begin{itemize}
\item {} 
\sphinxstyleliteralstrong{\sphinxupquote{sent}} \textendash{} a list of tokens in a single sentence

\item {} 
\sphinxstyleliteralstrong{\sphinxupquote{i}} (\sphinxstyleliteralemphasis{\sphinxupquote{int}}) \textendash{} position of a transformed word in a given sentence (token sequence)

\end{itemize}

\end{description}\end{quote}

\end{fulllineitems}


\end{fulllineitems}

\index{NER\_BiLSTM\_Glove\_i2b2 (class in ner\_plugins.NER\_BiLSTM\_Glove\_i2b2)@\spxentry{NER\_BiLSTM\_Glove\_i2b2}\spxextra{class in ner\_plugins.NER\_BiLSTM\_Glove\_i2b2}}

\begin{fulllineitems}
\phantomsection\label{\detokenize{contents:ner_plugins.NER_BiLSTM_Glove_i2b2.NER_BiLSTM_Glove_i2b2}}\pysigline{\sphinxbfcode{\sphinxupquote{class }}\sphinxcode{\sphinxupquote{ner\_plugins.NER\_BiLSTM\_Glove\_i2b2.}}\sphinxbfcode{\sphinxupquote{NER\_BiLSTM\_Glove\_i2b2}}}
Class that implements and performs named entity recognition using BiLSTM neural network architecture. The architecture uses GloVe
embeddings trained on common crawl dataset. Then the algorithm is trained on i2b2 2014 dataset.
\index{build\_tensor() (ner\_plugins.NER\_BiLSTM\_Glove\_i2b2.NER\_BiLSTM\_Glove\_i2b2 method)@\spxentry{build\_tensor()}\spxextra{ner\_plugins.NER\_BiLSTM\_Glove\_i2b2.NER\_BiLSTM\_Glove\_i2b2 method}}

\begin{fulllineitems}
\phantomsection\label{\detokenize{contents:ner_plugins.NER_BiLSTM_Glove_i2b2.NER_BiLSTM_Glove_i2b2.build_tensor}}\pysiglinewithargsret{\sphinxbfcode{\sphinxupquote{build\_tensor}}}{\emph{sequences}, \emph{numrecs}, \emph{word2index}, \emph{maxlen}, \emph{makecategorical=False}, \emph{num\_classes=0}, \emph{is\_label=False}}{}
Function to create tensors out of sequences
\begin{quote}\begin{description}
\item[{Parameters}] \leavevmode\begin{itemize}
\item {} 
\sphinxstyleliteralstrong{\sphinxupquote{sequences}} \textendash{} Sequences of words

\item {} 
\sphinxstyleliteralstrong{\sphinxupquote{numrecs}} \textendash{} size of the tensor

\item {} 
\sphinxstyleliteralstrong{\sphinxupquote{word2index}} \textendash{} mapping between words and its numerical representation (index). Loaded from file

\item {} 
\sphinxstyleliteralstrong{\sphinxupquote{maxlen}} \textendash{} Maximal lenght of the sequence

\item {} 
\sphinxstyleliteralstrong{\sphinxupquote{makecategorical}} \textendash{} Not used

\item {} 
\sphinxstyleliteralstrong{\sphinxupquote{num\_classes}} \textendash{} Not used

\item {} 
\sphinxstyleliteralstrong{\sphinxupquote{is\_label}} \textendash{} Not used, leave default for action performing

\end{itemize}

\item[{Returns}] \leavevmode


\end{description}\end{quote}

\end{fulllineitems}

\index{perform\_NER() (ner\_plugins.NER\_BiLSTM\_Glove\_i2b2.NER\_BiLSTM\_Glove\_i2b2 method)@\spxentry{perform\_NER()}\spxextra{ner\_plugins.NER\_BiLSTM\_Glove\_i2b2.NER\_BiLSTM\_Glove\_i2b2 method}}

\begin{fulllineitems}
\phantomsection\label{\detokenize{contents:ner_plugins.NER_BiLSTM_Glove_i2b2.NER_BiLSTM_Glove_i2b2.perform_NER}}\pysiglinewithargsret{\sphinxbfcode{\sphinxupquote{perform\_NER}}}{\emph{text}}{}
Function that perform BiLSTM-based NER
\begin{quote}\begin{description}
\item[{Parameters}] \leavevmode
\sphinxstyleliteralstrong{\sphinxupquote{text}} \textendash{} Text that should be analyzed and tagged

\item[{Returns}] \leavevmode
returns sequence of sequences with labels

\end{description}\end{quote}

\end{fulllineitems}


\end{fulllineitems}

\index{Mask\_abstract (class in masking\_plugins.Mask\_abstract)@\spxentry{Mask\_abstract}\spxextra{class in masking\_plugins.Mask\_abstract}}

\begin{fulllineitems}
\phantomsection\label{\detokenize{contents:masking_plugins.Mask_abstract.Mask_abstract}}\pysigline{\sphinxbfcode{\sphinxupquote{class }}\sphinxcode{\sphinxupquote{masking\_plugins.Mask\_abstract.}}\sphinxbfcode{\sphinxupquote{Mask\_abstract}}}
Abstract class that other masking plugins should implement
\index{mask() (masking\_plugins.Mask\_abstract.Mask\_abstract method)@\spxentry{mask()}\spxextra{masking\_plugins.Mask\_abstract.Mask\_abstract method}}

\begin{fulllineitems}
\phantomsection\label{\detokenize{contents:masking_plugins.Mask_abstract.Mask_abstract.mask}}\pysiglinewithargsret{\sphinxbfcode{\sphinxupquote{mask}}}{\emph{text\_to\_reduct}, \emph{context={[}{]}}, \emph{document={[}{]}}, \emph{replacement\_list=\{\}}}{}
Implementation of the method that should perform masking. Returns changed token
\begin{quote}\begin{description}
\item[{Parameters}] \leavevmode\begin{itemize}
\item {} 
\sphinxstyleliteralstrong{\sphinxupquote{text\_to\_reduct}} \textendash{} a token that should be changed

\item {} 
\sphinxstyleliteralstrong{\sphinxupquote{context}} \textendash{} a context around the token that should be reducted. It is a list of tokens. Can be sentence or more. It is optional variable and does not need to be used

\item {} 
\sphinxstyleliteralstrong{\sphinxupquote{document}} \textendash{} a whole document as a list of tokens. Can be used as context. Optional variable and does not need to be used

\item {} 
\sphinxstyleliteralstrong{\sphinxupquote{replacement\_list}} \textendash{} list of strings with their replacements. Can be used to search for tokens or part of it, in order to replace with the value of dictionary.

\end{itemize}

\end{description}\end{quote}

\end{fulllineitems}


\end{fulllineitems}

\index{Mask\_date\_simple (class in masking\_plugins.Mask\_date\_simple)@\spxentry{Mask\_date\_simple}\spxextra{class in masking\_plugins.Mask\_date\_simple}}

\begin{fulllineitems}
\phantomsection\label{\detokenize{contents:masking_plugins.Mask_date_simple.Mask_date_simple}}\pysigline{\sphinxbfcode{\sphinxupquote{class }}\sphinxcode{\sphinxupquote{masking\_plugins.Mask\_date\_simple.}}\sphinxbfcode{\sphinxupquote{Mask\_date\_simple}}}
Abstract class that other masking plugins should implement
\index{mask() (masking\_plugins.Mask\_date\_simple.Mask\_date\_simple method)@\spxentry{mask()}\spxextra{masking\_plugins.Mask\_date\_simple.Mask\_date\_simple method}}

\begin{fulllineitems}
\phantomsection\label{\detokenize{contents:masking_plugins.Mask_date_simple.Mask_date_simple.mask}}\pysiglinewithargsret{\sphinxbfcode{\sphinxupquote{mask}}}{\emph{text}}{}
Implementation of the method that should perform masking. Takes a token as input and returns a set string “DATE”

\end{fulllineitems}


\end{fulllineitems}



\chapter{Indices and tables}
\label{\detokenize{contents:indices-and-tables}}\begin{itemize}
\item {} 
\DUrole{xref,std,std-ref}{genindex}

\item {} 
\DUrole{xref,std,std-ref}{modindex}

\item {} 
\DUrole{xref,std,std-ref}{search}

\end{itemize}


\renewcommand{\indexname}{Python Module Index}
\begin{sphinxtheindex}
\let\bigletter\sphinxstyleindexlettergroup
\bigletter{m}
\item\relax\sphinxstyleindexentry{mask\_framework}\sphinxstyleindexpageref{contents:\detokenize{module-mask_framework}}
\item\relax\sphinxstyleindexentry{masking\_plugins}\sphinxstyleindexpageref{contents:\detokenize{module-masking_plugins}}
\indexspace
\bigletter{n}
\item\relax\sphinxstyleindexentry{ner\_plugins}\sphinxstyleindexpageref{contents:\detokenize{module-ner_plugins}}
\end{sphinxtheindex}

\renewcommand{\indexname}{Index}
\printindex
\end{document}